\documentclass[a4paper,12pt]{article}
\usepackage[utf8]{inputenc}
\usepackage{geometry}
\geometry{a4paper, margin=1in}
\usepackage{graphicx}
\usepackage{amsmath, amssymb}
\usepackage{hyperref}

\title{\textbf{Project Proposal: Swarm Intelligence in Distributed AI Inspired by Bees}}
\author{Andrea Grandi {}}
\date{}

\begin{document}

\maketitle

\section{Introduction}
Swarm intelligence is a decentralized and self-organizing approach to problem-solving inspired by the collective behavior of social organisms. This project aims to explore a distributed AI system inspired by the behavior of honeybees. The goal is to model and simulate their communication, resource allocation, and decision-making processes to develop a robust distributed AI framework.

\section{Background and Motivation}
Honeybees exhibit remarkable collective intelligence through behaviors such as foraging optimization, hive thermoregulation, and decision-making for nest selection. Their decentralized yet efficient coordination makes them an ideal inspiration for designing distributed AI systems. By leveraging these biological principles, this project seeks to create a novel swarm-based AI model applicable to domains such as robotics, logistics, and distributed computing.

\section{Objectives}
\begin{itemize}
    \item Develop an AI model based on honeybee swarm intelligence principles.
    \item Implement a distributed simulation where agents (representing bees) interact dynamically.
    \item Analyze emergent behaviors and optimize decision-making mechanisms.
    \item Compare the performance of the proposed model with traditional distributed AI approaches.
\end{itemize}

\section{Methodology}
\subsection{Data Collection \& Literature Review}
Study honeybee behavior, focusing on swarm coordination, foraging patterns, and decision-making strategies.

\subsection{Algorithm Design}
Develop a decentralized AI model inspired by bee communication methods such as the waggle dance and quorum sensing.

\subsection{Simulation \& Implementation}
Utilize a multi-agent simulation framework (NetLogo) to model bee interactions.

\subsection{Evaluation \& Optimization}
Measure system performance in terms of efficiency, robustness, and adaptability to dynamic environments.

\section{Expected Outcomes}
\begin{itemize}
    \item A functioning simulation demonstrating swarm intelligence in a distributed AI system.
    \item Insights into how bee-inspired coordination can improve decentralized problem-solving.
    \item A potential foundation for applications in distributed robotics, network optimization, and autonomous agent coordination.
\end{itemize}


\end{document}
